%%% Template File for Use with the hmcclinic.cls.
%%%
%%% C.M. Connelly <cmc@math.hmc.edu>
%%%  $Id: clinic-template.tex 368 2011-08-01 18:53:50Z cmc $
%%%
%%%  Tag: $Name$


%%% !!! HMC STUDENTS SHOULD REMOVE THE FOLLOWING COPYRIGHT NOTICE FROM
%%% !!! FINAL SUBMISSIONS.

%%% Copyright (C) 2004-2010 Department of Mathematics, Harvey Mudd College.

%%% See the COPYING document, which should accompany this
%%% distribution, for information about distribution and modification
%%% of the document and its components.

%%% !!! END COPYRIGHT NOTICE.

%%%%%%%%%%%%%%%%%%%%%%%%%%%%%%%%%%%%%%%%%%%%%%%%%%%%%%%%%%%%%%%%%%%%%%
%%% Note that for your Clinic report, you should remove any        %%%
%%% comments that aren't relevant to your report.  You should also %%%
%%% remove the copyright notice assigning copyright to the         %%%
%%% department (depending on your sponsor, you may need to assign  %%%
%%% copyright to the sponsoring organization.                      %%%
%%%%%%%%%%%%%%%%%%%%%%%%%%%%%%%%%%%%%%%%%%%%%%%%%%%%%%%%%%%%%%%%%%%%%%

%%% Preamble.

%%% The top part of your document is called the preamble.  You supply
%%% some basic information about the document (such as its title and
%%% author) in a form that LaTeX can understand here.


%%% The first active line in your LaTeX document is the \documentclass
%%% command, which loads a LaTeX class file.  Class files generally
%%% define the appearance of a document, and include a variety of
%%% structural commands.

%%% Clinic reports use the clinic class, which should be located
%%% somewhere in TeX's search path.

%%% For midyear reports, include the midyear option, as in
%%%   \documentclass[midyear]{hmcclinic}
\documentclass{hmcclinic}

%%% You can also load additional LaTeX packages, or style files, that
%%% affect the way that the document is laid out, typeset, or supply
%%% additional commands or environments.  If you choose to load
%%% additional packages, make sure that they appear *before* the
%%% line loading hyperref; hyperref changes pieces of other
%%% packages, so it's important that it be loaded last.

\usepackage{graphicx}           % More control over graphic inclusion.
% \usepackage{amsthm}             % AMS theorem styles
\setcounter{tocdepth}{2}
\usepackage{comment}
\usepackage{listings}


\usepackage{caption}
\lstset{language=C++}


% listings settings from http://en.wikibooks.org/wiki/LaTeX/Source_Code_Listings
\usepackage{color}

\definecolor{mygreen}{rgb}{0,0.6,0}
\definecolor{mygray}{rgb}{0.5,0.5,0.5}
\definecolor{mymauve}{rgb}{0.58,0,0.82}
\lstset{ %
  backgroundcolor=\color{white},   % choose the background color; you must add \usepackage{color} or \usepackage{xcolor}
  basicstyle=\footnotesize,        % the size of the fonts that are used for the code
  breakatwhitespace=false,         % sets if automatic breaks should only happen at whitespace
  breaklines=true,                 % sets automatic line breaking
  captionpos=b,                    % sets the caption-position to bottom
  commentstyle=\color{mygreen},    % comment style
  deletekeywords={...},            % if you want to delete keywords from the given language
  escapeinside={\%*}{*)},          % if you want to add LaTeX within your code
  extendedchars=true,              % lets you use non-ASCII characters; for 8-bits encodings only, does not work with UTF-8
  frame=single,                    % adds a frame around the code
  keepspaces=true,                 % keeps spaces in text, useful for keeping indentation of code (possibly needs columns=flexible)
  keywordstyle=\color{blue},       % keyword style
  language=Octave,                 % the language of the code
  otherkeywords={*,...},            % if you want to add more keywords to the set
  numbers=left,                    % where to put the line-numbers; possible values are (none, left, right)
  numbersep=5pt,                   % how far the line-numbers are from the code
  numberstyle=\tiny\color{mygray}, % the style that is used for the line-numbers
  rulecolor=\color{black},         % if not set, the frame-color may be changed on line-breaks within not-black text (e.g. comments (green here))
  showspaces=false,                % show spaces everywhere adding particular underscores; it overrides 'showstringspaces'
  showstringspaces=false,          % underline spaces within strings only
  showtabs=false,                  % show tabs within strings adding particular underscores
  stepnumber=2,                    % the step between two line-numbers. If it's 1, each line will be numbered
  stringstyle=\color{mymauve},     % string literal style
  tabsize=2,                       % sets default tabsize to 2 spaces
  title=\lstname                   % show the filename of files included with \lstinputlisting; also try caption instead of title
}





%%% Load all other packages before this point.

%%% Load hyperref.
\usepackage[breaklinks=true,
  bookmarks,
  pdfpagemode=UseOutlines,
  pdfpagelayout=SinglePage]{hyperref}


%%% The preamble can also be used to define your own commands and
%%% environments, set some constants that will be used throughout your
%%% document, and so on.

%%% As you may have guessed, LaTeX's comment character is the percent
%%% sign.  Any line that starts with a % will be ignored.  You can
%%% also use the comment character to add comments to the end of a
%%% line that will be parsed by TeX.

%%% The optional \includeonly command allows you to specify the names
%%% of chapters that you want to typeset.  It is useful for debugging
%%% or for working intensely on one particular part of your document
%%% when you don't want to take the time to retypeset the entire document.


%%% This optional command provides additional context around an error.
%%% It can be helpful when tracking down a problem. 
%\setcounter{errorcontextlines}{1000}


%%% Information about this document.

%%% I find it most useful to put identifying information about a
%%% document near the top of the preamble.  Technically, this
%%% information must precede the \maketitle command, which often
%%% appears immediately after the beginning of the document 
%%% environment.  Placing it near the top of the document makes it
%%% easier to identify the document, and keeps it from getting
%%% mixed up with the content of your document.

%%% So, some questions.

%% What is the name of the company or organization sponsoring your project?
\sponsor{Sandia National Laboratories}

%% What is the title of your report?
\title{Parallelizing Intrepid Tensor Contractions Using Kokkos}

%% Who are the authors of the report (your team members)?  (Separate
%% names with \and.)
\author{Brett Collins~(Project Manager) \and Alex Gruver \and Ellen Hui \and
Tyler Marklyn}

%% What is your faculty advisor's name?  (Again, separate names with
%% \and, if necessary.)
\advisor{Jeff Amelang}

%% Liaison's name or names?
\liaison{H. Carter Edwards}

%% By not specifying a date with the \date command, the date the
%% document is typeset will be added.

%% If you need to put in a specific date, do so with
%%  \date{May 13, 2004}
%% You probably shouldn't, however.

%%% End of information section.


%%% New commands and environments.

%%% You can define your own commands and environments here.  If you
%%% have a lot of material here, you might want to consider splitting
%%% the commands and environments into a separate ``style'' file that
%%% you load with \usepackage.

% \newcommand{\coolcommand}[1]{#1 is cool.} % Lets everyone know that
                                % the person or thing that you provide
                                % as the argument to the command is
                                % cool.

% \newcounter{cms}


%%% Some theorem-like command definitions.

%%% The \newtheorem command comes from the amsthm package.  That
%%% package is *not* loaded by the class file, so if you choose
%%% to use these commands, you'll need to load the package above.

% \newtheorem{thm}{Theorem}[chapter]
% \newtheorem{lem}{Lemma}[chapter]


%%% If you find that some words in your document are being hyphenated
%%% incorrectly, you can specify the correct hyphenation using the
%%% \hyphenation command.  Note that words are separated by
%%% whitespace, as shown below.

\hyphenation{ap-pen-dix wer-ther-i-an}


%%% The start of the document!

%% The document environment is the main environment in any LaTeX
%% document.  It contains other environments, as well as your text.

\begin{document}

%%% The front matter of a large document includes the title page or
%%% pages, tables of contents, lists of figures or tables, and so on,
%%% your abstract, a preface or introduction, and so on.  It's
%%% delineated with the \frontmatter command.

\frontmatter


%%% One of the things that the \frontmatter does is make page
%%% numbers appear as lowercase Roman numerals---i, vi, xii, and so
%%% on.

%%% The first thing in the front matter is your title page.  The title
%%% page is formatted by commands in the document class file, so you
%%% don't need to worry about what it looks like -- just putting the
%%% \maketitle command in your document (and filling in the necessary
%%% information for the identification commands above) is enough.

\maketitle


%%% Abstract

\begin{abstract}
  Your abstract should be a \emph{brief} summary of the contents of
  your report.  Don't go into excruciating detail here---there's
  plenty of room for that later.

  If possible, limit your abstract to a single paragraph, as your
  abstract may be used in promotional materials for the Clinic.
\end{abstract}


%%% Table of Contents, List of Figures, and List of Tables.
%%% 
%%% If you don't have any figures or tables in your report, you
%%% should comment out the appropriate command.  If you don't,
%%% you'll get an extra, mostly blank, page in your typeset report.

\tableofcontents
\listoffigures
\listoftables



%%% Acknowledgments.

\begin{acknowledgments}
%% Thank some people here, if you like.
\end{acknowledgments}

%%% End of the front matter.


%%% Beginnning of the main matter.

%% The main part of your report consists of normal, numbered
%% chapters as well as any appendices.  Bibliographies, indexes, and
%% so on are part of the back matter.  The main matter is opened with the
%% \mainmatter command.

\mainmatter


%%% Content.

%%% For smaller documents---especially those you're writing by
%%% yourself---you might write your entire report using a single LaTeX
%%% source file.  For larger documents, we recommend that you split
%%% the source file into several separate, smaller files.  The smaller
%%% files are ``included'' into your main, or ``master'' document
%%% using \include commands.

%%% Splitting your source has several advantages.  First, if you're
%%% working on a document with a group of people, it allows you to
%%% have more than one person working on different parts of the
%%% document at the same time (although we still recommend that you
%%% use CVS or a similar revision-control system!).  Second, smaller
%%% document chunks allow you to reorganize your document more
%%% easily.  If you later decide that Chapter 8 would be better as
%%% Chapter 4, all you have to do is swap the \include commands
%%% around.  For that reason, you should give your separate chapters
%%% meaningful names, such as ``introduction'', ``background'', or
%%% ``conclusions'' rather than calling them ``chapter1'',
%%% ``chapter2'', and so on.

%%% Finally, splitting the document allows you to concentrate on a
%%% particular section without being distracted by other
%%% sections---all you have to do is comment out the \include line for
%%% the sections you're not working on.  This technique can be
%%% especially useful when you're trying to track down a problem by
%%% allowing you to easily locate the file with the problem by
%%% ruling out the other sections.

%%% In our example document, we define several chapters that have
%%% useful information about writing Clinic or thesis reports or
%%% using LaTeX.  Here, we'll just use placeholders (but not
%%% chapter1, chapter2, etc!).  .


% Tyler's piece

% CHAPTER 1 -- Intro to everything
\chapter{Sandia National Laboratories}

\section{Background}

Sandia National Laboratories is a federally funded research and development
center owned and managed by the Sandia Corporation. The laboratory's primary
focus is the maintenance, management, and development of the United States'
nuclear arsenal. 

With the 1996 comprehensive nuclear test ban, Sandia began to
focus more heavily on computer simulations. These computationally intense
simulations have pushed Sandia to perform more and more optimizations on their
codebase, as faster-running programs allow greater throughput on simulation
results.

Traditionally, parallel algorithms used the Message Passing Interface (MPI)
standard, allowing many machines to work collaboratively on partitioned
subdomains of a global problem. The vast majority of scientific software
produced and used by the national labs relies on MPI to leverage both inter-node
and intra-node parallelism. In the case of intra-node parallelism, this model
pretends that the various computational engines within a computer are actually
separate computers. Until the present, this approach of using MPI across a
single node sacrificed some performance for the ease of a monolithic programming
model, and the performance penalty had not been high enough to motivate the
use of threads instead of processes. 

However, as the exascale push hits the power wall, interest has been growing in
the area of using higher performing and less power-hungry co-processors for the 
intra-node parallelism. Unfortunately, this means
that much existing code will have to be rewritten, as MPI cannot be used to
leverage the parallelism of co-processors such as Graphical Processing Units
(GPUs), which are further discussed in Section~\ref{CPU-GPU}.

\section{Problem}

The task of this clinic project has been to rewrite several kernels included in
libraries within Sandia's Trilinos Project. The Trilinos Project is a
collection of open source libraries intended for use in large-scale scientific
applications. Specifically, this clinic team aimed to rewrite some of these Trilinos kernels to be
efficient and thread-scalable on manycore architectures such as GPUs.

As well as presenting Sandia with a set of faster kernels, Sandia has also
requested that we present a general approach to parallelizing code. Many of the
mathematicians, engineers, and scientists at Sandia have little experience
writing code for GPUs or large thread count coprocessors such as Intel's Xeon Phi. Over
the course of our clinic project, we have uncovered a number of parallelization pitfalls 
along with techniques that work well for achieving speedup. By recording our experiences, we hope to
make it easier for Sandia's engineers to apply similar techniques to more of
their codebase after the termination of this clinic project.

The kernels we focused on during our project were in a tensor manipulation
library within Intrepid, a sub-package of Trilinos. The Intrepid package is a
library of kernels designed for performing discretizations of partial
differential equations. As such, by improving the performance of these kernels
within Intrepid, we can improve performance of any calculation-heavy simulation
libraries that rely on Intrepid to calculate tensor contractions. See
Chapter~\ref{sec:IntroIntrepid} for more detail on Intrepid.

\section{CPU vs GPU} \label{CPU-GPU}

Most traditional computer programs run on a Central Processing Unit (CPU). CPUs
are characterized by relatively low thread counts. For reference, a modern personal computer
typically supports 2-8 hardware threads. Even CPU nodes on scientific computing clusters usually
support fewer than 32 threads. Instead of supporting many hardware threads, CPU designers 
choose to dedicate a large portion of the chip for a CPU to
caching and other features that in some ways make up for
programming inefficiencies that would otherwise reduce performance.

Our project has mostly focused on writing code that will run on Graphical
Processing Units (GPUs). GPUs are characterized by extremely high thread counts\footnote{
In order to saturate its cores, a GPU requires a minimum of approximately 15,000 threads, 
with 150,000+ being required for good performance},
decreased memory per thread, and relatively small instruction sets when compared
to CPUs. In general, this means that programming on a GPU is less forgiving in the 
sense that sloppy programs will run significantly slower. For this reason,
despite the much higher level of parallelism afforded by the large thread
count, it is easy to write parallel GPU code that runs slower than
equivalent serial CPU code.

However, GPUs have many advantages when it comes to high performance computing.
Since GPUs have a smaller instruction set, they can devote more of their
transistors to arithmetic computation. This means that GPUs are capable of
executing significantly more floating point operations per second (FLOPS) than
CPUs. Additionally, GPUs are more efficient (in terms of FLOPs/watt) than CPUs, which makes them appealing
in supercomputers, where power consumption is a major concern.

Well implemented GPU code can yield significant speedup in certain
processing-heavy applications. Specifically, a problem may work well on a GPU if
it features high levels of arithmetic computation that can be calculated
independently; that is, if each calculation does not rely on the results of the other
calculations. For example, calculating a sequence of Fibonacci numbers is very
difficult for a GPU, as each number relies on results from the previous two numbers. A 
more ideal problem for a GPU would be to take two large arrays
and multiply them element-wise into a third array. When doing element-wise multiplication,
there is no reliance on previous results, which means that the threads on the GPU will
never have to wait for one another.

Another key consideration when writing high-performance code on any architecture
is the memory access pattern. Programs invariably need to retrieve data that is
stored in memory (RAM). On traditional CPU architectures, it is best to access
this memory sequentially, to take advantage of caching. A single thread of
execution on the CPU should ideally access memory locations immediately adjacent
to that same thread's previous access.

However, this is not the case on the GPU. On a GPU, groups of threads called
\emph{warps} all access memory at the same time. For this reason, among others, threads in a warp
should coordinate their memory accesses such that adjacent threads in
a warp access adjacent locations in memory, a pattern called 'coalescing'.

These differences between ideal CPU code and GPU code mean that code that is
written well for one architecture will usually perform poorly on the other if
ported directly without changing the data structures and memory access patterns.
Therefore, code tends to be architecture-dependent, so switching to a new
architecture often means rewriting an entire codebase. This is not ideal for
Sandia, where the mathematicians, engineers and scientists are interested in 
using the most up-to-date architecture without
needing to constantly overhaul their codebase.

\section{Kokkos}

In order to mitigate the effects of architecture dependent code overhauls,
Sandia is developing a C++ library called Kokkos, which is included in the
Trilinos package. Kokkos attempts to solve the issues of architecture dependence
by allowing programmers to write their code one time using Kokkos, and then
compile (or make other small tweaks) for optimization on a variety of
architectures. This is possible because the library helps manage the allocation
and access of memory across devices for the programmer. Kokkos also allows users
to write thread scalable software by providing an API for using fast,
non-locking data structures. Finally, Kokkos provides a concept of `thread
teams.' Thread teams are groups of threads that work together to solve a
problem, and can be used in nested parallelism algorithms such as those detailed
in Section~\ref{sec:reduction} and Section~\ref{sec:tiling}.

Ideally, all of Sandia's codebase would be written using Kokkos, so that no more
future large overhauls will be required as new architectures are released. For
this reason, all of the parallel code we have written for the clinic project has
used Kokkos.

The Kokkos package is still under development and no large-scale projects have
yet been fully implemented using Kokkos. As such, another function of this
clinic project has been to perform a large-scale test of Kokkos. Thus, we hope
to provide both a faster and more future-safe codebase, and also early feedback
on the Kokkos project. Sandia hopes for features from Kokkos to eventually be
included in the C++17 standard, so any feedback (both positive and negative) we
can provide would be useful.

Additionally, Sandia wishes to verify the claim that Kokkos is not slower than
other popular methods of thread parallelism for scientific computing, namely
OpenMP (for CPU parallelism) and Cuda (for GPU parallelism). In fact, Kokkos
uses OpenMP and Cuda backends, for compiling on CPUs and GPUs respectively. We
therefore hope to show that there is no overhead to using Kokkos rather than one
of the more established parallelism solutions.
          % tyler
    % Background
    % Problem
    % CPU vs GPU
    % Kokkos

% Ellen
\chapter{Intrepid}
\section{Tensor Contractions}
\section{Intrepid Contractions}
\section{Snippets}

        % ellen
    % Tensor Contractions in general
    % Intrepid specific contractions 
    % Serial snippets and descriptions

\chapter{Parallelism}

% Brett
\section{Flat Parallelism}

% Ellen
\section{Reduction}

% Alex
\section{Slicing}
Another general method we used for these contractions was Slicing. The first step of this method was to load one full contraction from the left matrix into shared memory. Then, we simultaneously computed every output element that was dependent on that contraction as input. The clearest way to explain the algorithm is by example. Consider one of the matrix multiplications in ContractFieldFieldScalar. 

\begin{figure}
    \centering
    \includegraphics[scale = .55]{ContractFieldFieldScalarGraphic}
    \caption{Demonstration of memory accesses for a slicing implementation of ContractFieldFieldScalar}
\end{figure}

	On the left, we have the first of the two input matrixes, who's first row elements are labeled $A-E$. On the right we have the second of the two inputs. For the sake of simplicity, assume that we have on block of five threads which are labeled by color. Each of the threads reads in one of the elements on the right and copies it into shared memory. In cases where the number of elements per contraction (row on the left) is unequal to the number of contractions (columns on the right), we set the number of threads per block equal to the number of contractions. This causes threads to either sit idle or loop multiple times when reading the elements on the left into shared memory, but this is clearly more efficient than forcing threads to compute more than one element.
	
	After the values of the contraction have been read into shared memory, we have each thread compute the output element corresponding to one contraction. This is shown on the right by the colored columns. Each thread reads every element from shared memory and computes the contraction by multiplying these elements with the columns of the right matrix. We see that throughout this progress, memory accesses will be coalesced within the block, since each thread reads the same element from shared memory then multiplies by an element that is adjacent to the other elements the rest of the block is reading at that time. 
	
	For every other block of threads, the approach is similar, if Figure 3.1 represents the first block of the contraction, then the second block will be represented as below. 

\begin{figure}[b]
    \centering
    \includegraphics[scale = .55]{ContractFieldFieldScalarGraphic2}
    \caption{Demonstration of memory accesses for the second block of a slicing implementation of ContractFieldFieldScalar}
\end{figure}

We see that for the FieldFieldScalar example, where our equation is given by $L_{C,\ell,P} \times R_{C, \mathcal{R}, P} = O_{C,\ell, \mathcal{R}}$, the number of blocks initialized by the algorithm will be equal $\ell \times C$, since there are $\ell$ blocks per matrix, and we have $C$ matrices. Additionally, there will be $\mathcal{R}$ threads per block. 

	Code for executing the algorithm as described above is included below, although it has been simplified for clarity. 

\begin{figure}[ht]
    \begin{lstlisting} [basicstyle=\tiny]
     extern __shared__ float sliceStorage[];
     const unsigned int col = threadIdx.x;
     unsigned int currentBlock = blockIdx.x;
     unsigned int numBlocks = numBasis*numCells;
     
     syncthreads();
     const unsigned int cell = currentBlock / numBasis;
     const unsigned int row = currentBlock - cell * numBasis;

     for (unsigned int p = threadIdx.x; p < contractionSize; p += blockDim.x) {
        sliceStorage[p] = dev_contractionData_Left[cell*numBasis*contractionSize + row*contractionSize + p];
     }
     syncthreads();

     float sum = 0;
     for (int p = 0; p < contractionSize; ++p) {
       sum += sliceStorage[p] * dev_contractionData_Right[cell*numBasis*contractionSize +
        p*numBasis + col];
     }

     dev_contractionResults[cell*numBasis*numBasis + row*numBasis + col] = sum;
 \end{lstlisting}
\caption{Code from slicing algorithm on \texttt{ContractFieldFieldScalar}
\label{lst:ContractFieldFieldScalarSlice}} 
\end{figure}

	The main advantage of this approach is that it is easily generalizable to tensor contractions of higher dimensions. Unlike tiling, which is significantly less intuitive in higher dimensions, it is easy to implement slicing in higher dimensions by loading a larger slice into shared memory. Because of its reliance on shared memory, there are many use cases in which we would expect slicing to perform poorly. Intuitively, slicing is reliant on large contraction sizes to produce speedup because in situations where the number of threads per block is low it is unable to saturate the GPU. While this problem can be remedied by increasing the number of contractions per block, it can introduce problems with shared memory. Since shared memory is limited by nature, slicing has to balance the amount of work per block with the amount of shared memory available to that block.
	
	In situations where the problem has an inherently large amount of reuse like ContractFieldFieldScalar, this problem can be remedied to some degree, but in contractions without this feature, like ContractDataDataScalar, it seems clear that slicing will not be an efficient algorithm. 
	
When we compare slicing approaches using one contraction per block to independent flat parallelism on promising problems, we get underwhelming results. 

    \includegraphics[scale = .17]{slicingvsindependent}

These results were generated by comparing independent algorithms to slicing on ContractFieldFieldScalar with $\ell = \mathcal{R} = 10, P = 8 - 1024$. We see that in the corner where the memory size is small and contraction size is small we get a small amount of speedup relative to independent Cuda code, which is promising. This is the corner where we would expect slicing to perform the best in comparison to independent parallelism, since in this corner flat parallelism is unable to fully saturate the GPU. The benefits of reuse in this corner are significant enough to outcompete flat parallelism. On the rest of the graph, however, the inability of slicing to saturate the GPU means that it is significantly slower that flat parallelism. Since $\ell = \mathcal{R} = 10$, the algorithm naturally only spawns 10 threads per block, which is not enough to produce good results. 

On problems with larger basis functions we see better results for slicing. For example, consider the following use case of ContractFieldFieldTensor: $\ell = \mathcal{R} = 125$, $P = 216$, $t_1 = t_2 = 3$.

\vspace{10mm}
the data exists for this here: https://github.com/Sandia2014/kokkos-intrepid/tree/slicing+tiling/ContractFieldFieldTensor, will make graph later


We see that while slicing performs better, it is still eclipsed by independent cuda. 

We're still working on slicing that does two rows per block so hopefully that will be done soon and we'll have data for it. 

% Alex
\section{Tiling}

The final parallelization technique we used for these tensor contractions was tiling. This technique is similar to the tiled technique for matrix multiplication used in serial operations. Instead of relying on the cache to retain the relevant pieces of information, however, we use shared memory to explicitly store the data we care about. Once again, we will explain this algorithm by example. Consider one of the matrix multiplications in ContractFieldFieldScalar shown above. 

\begin{figure}
    \centering
    \includegraphics[scale = .7]{ContractFieldFieldScalarGraphicTiling}
    \caption{Demonstration of memory accesses for a tiling implementation of ContractFieldFieldScalar}
\end{figure}

For the sake of simplicity we'll consider a block to be four threads, which simplifies our computation since the matrix is four by four. On the left hand side, the block loads a four element tile into the shared memory of the threads. Once these elements are loaded into memory, each thread can begin computation of their element in the output matrix. Each thread computes as much of their output element as they can using the elements in shared memory, then we load a new tile into shared memory and continue the process, as shown below. We see that in this case we will have to load two tiles into shared memory before we have computed every output element in its entirety. 

\begin{figure}
    \centering
    \includegraphics[scale = .7]{ContractFieldFieldScalarGraphicTiling2}
    \caption{Demonstration of memory accesses for a tiling implementation of ContractFieldFieldScalar}
\end{figure}

	Tiling can be viewed as a more specialized version of slicing, since they both use similar access patterns for shared memory. The difference between the two lies in tilings usage of multiple contractions per block, as well as the the distribution of a contractions operations over multiple loops of the routine. Because of these differences, tiling can routinely saturate the GPU in a way that pure slicing cannot, since the algorithm inherently limits the shared memory usage per block by reusing the same shared memory multiple times. Additionally, if we set the dimension of our tiles intelligently, we can reliably saturate the GPU with both blocks and threads, something that is very difficult to do adaptively with pure slicing. 
	
	Unfortunately, it is much less clear how exactly to tile in multiple dimensions. Unlike slicing, there seem to be multiple distinct ways of approaching the problem. One could create "tiles" with dimension equal to the contraction size, or any number less than the contraction dimension by unrolling the contraction to some intermediate degree. We haven't been able to fully explore every possibility in this area, and have simply treated the higher dimensional contractions as a fully unrolled contraction of one dimension. It is possible, however, that in some situations it would be more effective to create tiles with multiple degree. These tiles would have a different layout in memory who's efficiency would vary by situation. 
	
	Excerpts from our Cuda implementation of tiling are included below. The code assumes that tileSize (the horizontal and vertical dimensions of a tile) evenly divides both the contraction size and $\ell = \mathcal{R} = \text{numBasis}$.
	
\begin{figure}
    \begin{lstlisting} [basicstyle=\tiny]
  extern __shared__ float tileStorage[];
  const unsigned int numbersPerTile = tileSize * tileSize;
  const unsigned int numberOfHorizontalTiles = contractionSize / tileSize;
  const unsigned int numberOfVerticalTiles = numBasis / tileSize;

  const unsigned int numberOfTiles = numCells * numberOfVerticalTiles * numberOfVerticalTiles;

  const unsigned int subRow = threadIdx.x / tileSize;
  const unsigned int subCol = threadIdx.x  - subRow * tileSize;

  unsigned int resultTileIndex = blockIdx.x;

  unsigned int resultSubmatrixIndex = resultTileIndex % (numberOfVerticalTiles * numberOfVerticalTiles);
  unsigned int resultMatrix = resultTileIndex / (numberOfVerticalTiles * numberOfVerticalTiles);

  // for tileNumber in 0...numberOfTilesPerSide
  for (unsigned int tileNumber = 0; tileNumber < numberOfHorizontalTiles; ++tileNumber) {
      
      // calculate result tile indices
      const unsigned int resultTileRow = resultSubmatrixIndex / numberOfHorizontalTiles;
      const unsigned int resultTileCol = resultSubmatrixIndex  -
        resultTileRow * numberOfHorizontalTiles;

      // calculate this threads actual output index
      const unsigned int row = resultTileRow * tileSize + subRow;
      const unsigned int col = resultTileCol * tileSize + subCol;

      // these are base indices into the shared memory
      const unsigned int leftBaseIndex = subRow * tileSize;
      const unsigned int rightBaseIndex = numbersPerTile + subCol;

      const unsigned int resultIndex = row * numBasis + col;

      // load the left and right tiles into shared memory
      syncthreads();
      tileStorage[threadIdx.x] = dev_contractionData_Left[resultMatrix * numBasis * contractionSize
        + row * contractionSize + tileNumber * tileSize + subCol];
      tileStorage[threadIdx.x + blockDim.x] = dev_contractionData_Right[resultMatrix * numBasis * contractionSize
        + (tileNumber * tileSize + subRow) * numBasis + col];
      
      // make sure everyone's finished loading their pieces of the tiles
      syncthreads();
      double sum = 0;
      for (unsigned int dummy = 0; dummy < tileSize; ++dummy) {
        sum +=
          tileStorage[leftBaseIndex + dummy] *
          tileStorage[rightBaseIndex + dummy * tileSize];
      }
      dev_contractionResults[resultIndex] += sum;
    }

 \end{lstlisting}
\caption{Code from tiling algorithm on \texttt{ContractFieldFieldScalar}
\label{lst:ContractFieldFieldScalarSlice}} 
\end{figure}


Thus far in our research, we have found tiling to be the most effective algorithm for realizing parallel speedup. 

\begin{figure}
    \centering
\includegraphics[scale = .2]{tilinguc1}
\end{figure}
Consider the graph generated above for the following use case of ContractFieldFieldScalar, $\ell = \mathcal{R} = 125$, $P = 216$. We see that Tiling outperforms both flat parallelism and team reductions across the board. This trend continues for smaller use cases as well, as shown below when $\ell = \mathcal{R} = 8$, $P = 8$.

\begin{figure}[h]
    \centering
\includegraphics[scale = .2]{tilinguc2}
\end{figure}

In general, we have found that 2D tiling is the most effective method for achieving parallel speedup on these kernels. While there may be some potential for exploration of higher dimensionality tiles, it seems doubtful that these layouts will be able to accomplish significantly more speedup. 
           % brett, ellen, alex
    % Flat parallelism (Brett)
    % Reduction (Ellen)
    % Slicing (Alex)
    % Tiling (Alex)

% Brett
\chapter{Experience with Kokkos}
\section{Performance}
Since Kokkos uses Cuda and OpenMP as a backend to achieve faster performance, we
chose to do some testing to confirm that Kokkos performs as well as these two
solutions. If Kokkos performed worse then Cuda or OpenMP, then programmers might
prefer these other solutions instead.  Fortunately, we found that Kokkos matches
the performance of Cuda and OpenMP in almost all cases.  The rest of this
section will describe our strategy for testing the performance of Kokkos
compared to Cuda and OpenMP, present graphs showing the differences observed,
and analyze the graphs.

In order to compare Kokkos, Cuda, and OpenMP, we wrote algorithmically
equivalent code using all three paradigms (using the same data layout and memory
access pattern) and recorded the runtime of each version.  To reduce noise in
the timing data, we repeated the same calculations five times and used the
average time.

Note that because we were unsure how Kokkos implements the team\_reduce()
function, we could not write a matching Cuda reduction.  Based on our project
priorities, we chose not to pursue this further.

These graphs present some of the performance differences and similarities of
Kokkos, Cuda, and OpenMP. Figure~\ref{fig:ContractDataDataScalar Kokkos
performance comparison} shows the raw times of Kokkos Cuda, Cuda, Kokkos OpenMP,
and OpenMP for ContractDataDataScalar.

\begin{figure}[!ht]
{\includegraphics[scale=.4]{CDDS_RawTimes_2d_largestSize_Comparison.pdf}}
\caption[ContractDataDataScalar Kokkos performance comparison]{
    Performance of Kokkos Cuda, Cuda, Kokkos OpenMP,
and OpenMP for ContractDataDataScalar with a memory size of 1 GB.}
\label{fig:ContractDataDataScalar Kokkos performance comparison}
\end{figure}

In this graph, the y-axis is time in seconds, so closer to zero is better. The
x-axis plots different contraction sizes.  Here, Kokkos OpenMP and OpenMP are
almost perfectly overlapping. We are not quite sure why they are not perfectly
overlapping, but it appears too consistent to be random noise. However, the
difference is small enough as to be fairly insignificant. 

Kokkos Cuda, however, shows major differences compared to Cuda. The two perform
identically for the smaller problems but diverge by a significant amount for
bigger problems. This trend exists because Kokkos launches a different number of
blocks compared Cuda; Kokkos launches fewer blocks, with the intention of
reusing them.  We believe this doesn't affect small problem sizes because such
problems require fewer blocks than the large problem sizes, so both Kokkos and
Cuda launch enough blocks.  However, there is clearly a difference in the bigger
problem sizes. 

Figure~\ref{fig:cffscomparison} shows ContractFieldFieldScalar with the slicing
technique (which uses shared memory) for both Kokkos Cuda and Cuda.  It also
includes the flat parallel algorithm in both Kokkos Cuda and Cuda.

\begin{figure}[!ht]
{\includegraphics[scale=.4]{CFFS_RawTimes_2d_largest_Comparison.pdf}}
\caption[ContractFieldFieldScalar Kokkos performance comparison]{
    Performance of the ``slicing'' nested parallelism approach.}
\label{fig:cffscomparison}
\end{figure}

In Figure~\ref{fig:cffscomparison}, the Cuda slicing performance is almost
identical to the Kokkos slicing performance. This shows that Kokkos's use of
shared memory matches that of Cuda.

Overall, most of our graphs show that Kokkos performs almost identically to Cuda
and to OpenMP.  We therefore accepted that Kokkos is not adding any major
overhead. There may be slight differences due to optimization choices, but
Kokkos performs similarly to the other multithreading solutions.

\section{Code Snippets}
Another major factor that plays into whether programmers will use a language,
feature, or library is code complexity and ease of coding. Thus, we also
investigated the usability, readability, and intuitiveness of Kokkos compared to
Cuda and OpenMP.

Parallelizing code with Kokkosis significantly more complex than doing the same
with OpenMP.  However, OpenMP is considerably less flexible than Kokkos; it
works only on the CPU, and does not generalize to the GPU.  Therefore, in the
case of code that needs only to run on the CPU, we would strongly advise OpenMP
because of its simplicity.  However, that is not the niche that Kokkos is trying
to fill. 

Cuda, however, requires similar amounts of code to Kokkos.  The code snippets
presented will point out the differences and similarities directly.  First, we
will show the data setup step of moving data onto the GPU, and then move to
comparing and contrasting the Cuda kernel and the Kokkos functor.

Figure~\ref{lst:ContractFieldFieldScalar Cuda Data Setup} shows the setup of the data on the GPU for Cuda.

\begin{figure}[!htb]
	\begin{lstlisting}
float * dev_leftDataArray;

cudaMalloc((void **) &dev_leftDataArray, 
	numContractions * numLeftFields * numPoints * 
	sizeof(float));
	
cudaMemcpy(dev_leftDataArray, &leftDataArray[0], 
	numContractions * numLeftFields * numPoints * sizeof(float), 
	cudaMemcpyHostToDevice);
	\end{lstlisting}

\caption{Code from Cuda \texttt{ContractFieldFieldScalar}
\label{lst:ContractFieldFieldScalar Cuda Data Setup}}
\end{figure}

There are three steps in the process: declaring a pointer to the data on the
CPU, allocating an array with the correct size on the GPU, then copying the data
over to the GPU from CPU (host) memory. This process is relatively simple and
self-explanatory.

Equivalent Kokkos code is shown in Figure~\ref{lst:ContractFieldFieldScalar
Kokkos Cuda Data Setup}

\begin{figure}[!htb]
	\begin{lstlisting}
typedef Kokkos::Cuda	DeviceType;
typedef Kokkos::View<float***, Kokkos::LayoutRight, DeviceType>
	ContractionData;
typedef typename ContractionData::HostMirror
	ContractionData_Host;

ContractionData dev_ContractData_Left("left_data",
	numContractions,
	numLeftFields,
	numPoints);

ContractionData_Host contractionData_Left = 
	Kokkos::create_mirror_view(dev_ContractData_Left);

for (int cell = 0; cell < numContractions; ++cell) {
	for (int lbf = 0; lbf < numLeftFields; ++lbf) {
		for (int qp = 0; qp < numLeftFields; ++qp) {
			contractionData_Left(cell, lbf, qp) = 
				contractionDataLeft[cell*numLeftFields*
				numPoints + lbf*numLeftFields + qp];
		}
	}
}
	\end{lstlisting}
\caption{Code from Kokkos Cuda \texttt{ContractFieldFieldScalar}
\label{lst:ContractFieldFieldScalar Kokkos Cuda Data Setup}}
\end{figure}

The Kokkos code first defines and creates the device and host Views. One of the
major differences compared to Cuda is that Kokkos uses its own data structure, a
View, instead of an array. This requires typedefs to define the
Views, but the small amount of extra work gives the
programmer much more control over the data. The control also comes at the cost
of having to use loops to copy the data into the host view instead of simply
copying raw memory.

However, this initial work is done only once, and allows the user to change the
layout of the data by simply changing the Kokkos::LayoutRight to
Kokkos::LayoutLeft.  This is useful in optimizing the data layout for both the
CPU and GPU. Overall, Kokkos is more verbose, but also more abstract, as it must
perform on both the CPU and GPU while Cuda only runs on the GPU. 

In a program's computational portion, Cuda uses a kernel while Kokkos uses a
functor.  However, for programs doing the same calculation, the parenthesis
operator in a Kokkos functor is almost an exact replica of the code in
the corresponding Cuda kernel. A Cuda kernel for ContractFieldFieldScalar is
shown in Figure~\ref{lst:ContractFieldFieldScalar Cuda kernel}.

\begin{figure}[htb]
	\begin{lstlisting}
__global__ void
cudaContractFieldFieldScalar_Flat_kernel(int numContractions,
	int numLeftFields,
	int numRightFields,
	int numPoints,
	float * __restrict__ dev_contractData_Left,
	float * __restrict__ dev_contractData_Right,
	float * dev_contractResults) {
	int contractionIndex = blockId.x * blockDim.x + threadIdx.x;
	while (contractionIndex < numContractions) {
		int myID = contractionIndex;
		int myCell = myID / (numLeftFields * numRightFields);
		int matrixIndex = myID % (numLeftFields * 
			numRightFields);
		int matrixRow = matrixIndex / numRightFields;
		int matrixCol = matrixIndex % numRightFields;
		
		// Calculate now to save computation later
		int lCell = myMatrix * numLeftFields * numPoints;
		int rCell = myMatrix * numRightFields * numPoints;
		int resultCell = myMatrix * numLeftFields * 
			numRightFields;
		
		float temp = 0;
		for (int qp =0; qp < contractionSize; qp++) {
			temp += dev_contractData_Left[lCell + 
				qp*numLeftFields + matrixRow] *
				dev_contractData_Right[rCell + 
				qp*numRightFields + matrixCol];
		}

		dev_contractResults[resultCell + 
			matrixRow * numRightFields + matrixCol] = 
				temp;
		
		contractionIndex += blockDim.x * gridDim.x;
	}
}
	
	\end{lstlisting}
\caption{Code from Cuda \texttt{ContractFieldFieldScalar}
\label{lst:ContractFieldFieldScalar Cuda kernel}}
\end{figure}

The parenthesis operator in the corresponding Kokkos functor is shown in
Figure~\ref{lst:ContractFieldFieldScalar Kokkos Cuda functor}.

\begin{figure}[htb]
	\begin{lstlisting}
KOKKOS_INLINE_FUNCTION
void operator() (const unsigned int elementIndex) const {
	int myID = elementIndex;
	int myCell = myID / (_numLeftFields * _numRightFields);
	int matrixIndex = myID % (_numLeftFields * _numRightFields);
	int matrixRow = matrixIndex / _numRightFields;
	int matrixCol = matrixIndex % _numRightFields;

	float temp = 0;
	for (int qp = 0; qp < _numPoints; qp++) {
		temp += _leftFields(myCell, qp, matrixRow) *
			_rightFields(myCell, qp, matrixCol);
	}
	_outputFields(myCell, matrixRow, matrixCol) = temp;
}
	\end{lstlisting}
\caption{Code from Kokkos Cuda \texttt{ContractFieldFieldScalar}
\label{lst:ContractFieldFieldScalar Kokkos Cuda functor}}
\end{figure}

Although there are more lines of code in the Kokkos functor (the code required to
declare the data members and the constructor), the Kokkos code is readable and
uncluttered.

The Kokkos functor does not need to calculate each thread's ID, while the Cuda
kernel has to use built-in constants such as blockId.x and  blockDim.x.
Indexing into a View is easier than indexing into a primitive C-style array,
especially when changing the layout of the data between LayoutLeft and
LayoutRight because no code changes need to occur in the functor.

\section{Personal Experience and Thoughts}
A task of the project was to document our experiences and thoughts about Kokkos,
including any issues that we have run into. Using new tools and learning new
syntax always has its tough periods, and getting used to Kokkos definitely had
some periods where we had no idea why a program was not compile or giving an
incorrect answer (especially in the beginning). But, after the initial learning
curve everything seemed to flow pretty well and make sense. 

Our team has never actually been responsible for installing Kokkos on our
machine, instead our liaison, Dr. Carter Edwards, did that for us. We are unable
to talk about the difficulties of downloading and installing the Kokkos library
on our machine, but we did have lots of trouble trying to compile and linking
against Kokkos originally. This was due to the fact that the same flags need to
be used when installing and compiling and linking against Kokkos. However, since
we did not install Kokkos ourselves and the documentation showing how to compile
and link against Kokkos used different flags than what were used during our
installation, we struggled for a while. Already this shows how Kokkos'
documentation is not as developed as one would like, which we will bring up
later, but it is understandable since Kokkos is new. 

Another obstacle that slowed us down when first using, is Kokkos' use of magic
words. For example, Kokkos requires the programmer to typedef Kokkos::Cuda or
Kokkos::OpenMP to device\_type, and it must be device\_type, not some other
name. Although the programmer can easily fix this, if the programmer is unaware
of this requirement it can cause a lot of hassle for a while. Every team member
ran into this at one time or another, but after a while we got used to it. When
following examples we learned to use the same names for the typedefs to make
sure that we did not run into another bug with the same nature. Once again
documentation would have helped in this situation, but there is not much
documentation, all we have are examples. On the bright side however, since we
were able to write all of our programs by simply following a few examples we
were able to see some of Kokkos' intuitiveness. Overall we really enjoy Kokkos'
philosophy and structure, which as mentioned before, is almost identical to
Intel's Thread Building Blocks (TBB). If you are familiar with TBB then learning
Kokkos is almost as simple as learning the syntax because they are in the same
paradigm. 

As previously mentioned, Kokkos has very little documentation. For any emerging
technology it is understandable that the creators choose to focus on
functionality instead of documentation, but the documentation needs to catch up
at some point. The examples were very helpful in getting us to our end goal of
working code, but examples are not as helpful in understanding what exactly is
happening, the meaning behind some portions of code, or why certain code is
necessary. Documentation would have also been helpful in seeing the default
values for functions and Views, as well as the other arguments that could have
been passed instead. There were many times we tried to use Google to find
information about Kokkos, but many times the information would point to
uncommented pieces of code, which is not always helpful in determining what is
going on. Overall we believe the documentation for Kokkos needs to improve in
order for new users to get past the initial learning curve and spread the word
about Kokkos. 

As a whole, our team's experience with Kokkos has been positive and see that it
offers a great alternative to other solutions that allow multithreading on
multiple architectures. A quick overview of the benefits of using Kokkos are:
Kokkos can create multithreaded code on the CPU, GPU, and XeonPhi, Views can
easily change the layout of the data, functors seem to keep the code cleaner and
more readable than Cuda's kernels, and the fact that Kokkos is a C++ library and
not a a new language adds simplicity. Some of the downsides and changes that we
believe would improve Kokkos include Views having more layouts than LayoutRight
and LayoutLeft, the use of magic words (or lack of using the right magic words)
can create bugs that are hard to find, the example code should include comments
to describe what is happening, and finally the documentation needs to improve.
However, extended use of Kokkos will solve most of these problems except for
Views being limited to two layout types, which is why our team had an overall
good experience with Kokkos.
    % brett
    % Performance 
    % Code snippet

% Tyler

% Expected vs. Actual Performance of this clinic project
\chapter{Our Performance}

The original goal of this clinic project was to parallelize a number of tensor
manipulation kernels in the Intrepid library, and then move on to other kernels
in a different Trilinos library.  However, we have instead focused only on the
tensor manipulation kernels, without having moved on to any other kernels. The
reason for this is twofold:

Firstly, we underestimated the number of obstacles we would encounter over the
course of our project.  Before this project, no member of our team had written
performance-oriented parallel code, forcing us to spend much of the first
semester learning the concepts, techniques, and languages of the field.
However, even as we grew more familiar with parallelizing high-performance code,
we also ran into a number of issues with Kokkos.  Because Kokkos is still a
young project, at the beginning of the semester it had very little
documentation, as well as a few other issues that gave us difficulties.  Over
winter break, we received a newly updated version of Kokkos that included fixes
for some of these issues, along with some new features.

The second reason we did not move on to another package was the Kokkos update we
received over winter break.  The update introduced new functionality, allowing
us to write team-oriented parallel code and implement algorithms such as team
reductions.  The new Kokkos package also gave us access to shared memory on the
GPU. At this point, rather than merely writing flat parallel versions of other
kernels, we decided to focus more heavily on general parallelization techniques
using Kokkos teams for tensor contractions, taking the new Kokkos features into
account.  Thus, because we chose to focus on team techniques, we spent the
spring semester implementing, testing, and plotting results from from various
algorithms as applied to the tensor manipulations library instead of moving to a
different library.

Overall, although our goals diverged from our initial goals as laid out in our 
statement of work, we accomplished a significant amount of exploratory work this
semester. We believe that the work we have acoomplished will be useful to Sandia 
National Laboratories as they continue along the path of making their codebase
thread-scalable.    % tyler


%%% Appendices.

%%% Appendices are just like chapters, only they're generally
%%% lettered rather than numbered (although that depends on your
%%% document class, of course).

%%% The appendices are delineated with the \appendix command.
%%% Individual appendices are begun with the standard \chapter or
%%% \section commands.  In our example, we'll \include them just as
%%% we did other chapters.

%%% If you don't have any appendices, comment out the \appendix
%%% command.

\appendix

\include{our_appendix}

\include{our_source_code}


%%% Back matter.

%%% The back matter of a document is where the bibliography, index,
%%% glossary, and other unnumbered chapters or sections occur.  It
%%% starts, not surprisingly, with the \backmatter command.

\backmatter


%%% Bibliography.

%%% BibTeX is the tool to use for citations and layout of your
%%% bibliography.  Instead of having to type ``[5]'' or ``(Jones,
%%% 1968)'' (and keep track of which citation is which and renumber
%%% them as you add more references to your bibliography), you use
%%% special commands that allow BibTeX and LaTeX to automatically put
%%% the correct information in the right place.

%%% Depending on your field, it may or may not be appropriate to list
%%% references for which you haven't included specific citations.  If
%%% your field sanctions such practices, or if you just want to get an
%%% idea of what you have in your bibliography file, you can include
%%% everything with the \nocite{*} command.
\nocite{*} 


%%% The appearance of your bibliography and citations in your text are
%%% defined by a combination of any bibliography-related LaTeX
%%% packages (such as natbib, harvard, or chicago) and the particular
%%% bibliography style file that you load with the \bibliographystyle
%%% command.  Bibliography-style files end in .bst; you can find them
%%% by searching your file system using whatever tools you have for
%%% doing searches.  (On most modern Unices, ``locate .bst'' will give
%%% you an idea of what's available.)

% Standard bibliography style.
\bibliographystyle{hmcmath}

% Annotated bibliography style.
% \bibliographystyle{hmcmathannote}


%%% The particular bibliography data file or files that you want to
%%% use are specified with the \bibliography file.  Multiple files are
%%% separated by commas.

%%% You might want to use multiple bibliography (or ``bib'') files if
%%% you had a master bib file containing references you use again and
%%% again, and another containing only records for references for a
%%% particular project.

%%% Many people create a single, large bib file that they use for
%%% everything they write.  That approach requires you to \cite every
%%% reference that you want to use in your document -- using
%%% \nocite{*} with a huge bibliography database will give you a large
%%% bibliography containing many references you haven't consulted for
%%% your particular document!

\bibliography{our_bib_file}


%%% Glossary or Index.

%%% If you were going to include a glossary or index in your document,
%%% the relevant commands would appear here.

%%% If you think that you would like to include such features, talk
%%% with someone who's worked with LaTeX a lot very early in your
%%% writing process.  These commands require you to do a bit of
%%% thinking about what you would want to index or gloss in
%%% advance---going back though a completed document to add \index
%%% commands is *not fun*.


\end{document}


