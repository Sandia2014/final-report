% Tyler

% Expected vs. Actual Performance of this clinic project
\chapter{Our Performance}

\textit{\textbf{Note to current readers:} The project is still in progress. Our code freeze does not occur
until Friday, April 17, with data collection potentially occurring after that. Therefore, this section remains 
largely incomplete and unedited in this draft, and will be completed at a future date.}

The original stated goal of this clinic project was to parallelize a number of tensor manipulation 
kernels in the Intrepid library, and then move on to other kernels that performed more complex 
computations. However, by the end of our term as a clinic team, we will have only focused our 
energy on the tensor manipulaiton kernels, without having moved on to any other kernels. The 
reason for this is twofold:

Firstly, we underestimated the number of obstacles we would encounter over the course of our project. 
Originally, many of these obstacles stemmed from the fact that no member of our team had written 
performance-oriented parellel code before beginning work on the project. However, even as we grew more 
familiar with the concepts involved, we also ran into a number of issues with Kokkos, which is effectively 
still in an alpha testing stage, and at the beginning of the semester had very little documentation, and 
a number of other issues that made it difficult to work with. Over winter break, we received a newly updated 
version of Kokkos that included a few fixes for some of the issues we had been having, along with some new 
features.

The second reason we never made it to another package was the Kokkos update we received over winter break.
With the update, we could write team-oriented parallel code, which allowed us to implement algorithms such as
team reductions, and it allowed us access to shared memory on the GPU. At this point, rather than merely writing 
flat parallel versions of a larger number of the kernels, we decided to focus more heavily on general parallelization
techniques using Kokkos teams for tensor contractions in order to find the best way to parallelize our desired kernels.
As a result of our shift to focusing on team techniques, we spend the entire spring semester working on 
implementing, testing, and plotting results from from various algorithms as applied to the tensor manipulations library, 
and never moved on to a second kernel (as we had originally intended to do).
