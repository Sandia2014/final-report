% Tyler

% Expected vs. Actual Performance of this clinic project
\chapter{Our Performance}

The original goal of this clinic project was to parallelize a number of tensor
manipulation kernels in the Intrepid library, and then move on to other kernels
in a different Trilinos library.  However, we have instead focused only on the
tensor manipulation kernels, without having moved on to any other kernels. The
reason for this is twofold:

Firstly, we underestimated the number of obstacles we would encounter over the
course of our project.  Before this project, no member of our team had written
performance-oriented parallel code, forcing us to spend much of the first
semester learning the concepts, techniques, and languages of the field.
However, even as we grew more familiar with parallelizing high-performance code,
we also ran into a number of issues with Kokkos.  Because Kokkos is still a
young project, at the beginning of the semester it had very little
documentation, as well as a few other issues that gave us difficulties.  Over
winter break, we received a newly updated version of Kokkos that included fixes
for some of these issues, along with some new features.

The second reason we did not move on to another package was the Kokkos update we
received over winter break.  The update introduced new functionality, allowing
us to write team-oriented parallel code and implement algorithms such as team
reductions.  The new Kokkos package also gave us access to shared memory on the
GPU. At this point, rather than merely writing flat parallel versions of other
kernels, we decided to focus more heavily on general parallelization techniques
using Kokkos teams for tensor contractions, taking the new Kokkos features into
account.  Thus, because we chose to focus on team techniques, we spent the
spring semester implementing, testing, and plotting results from from various
algorithms as applied to the tensor manipulations library instead of moving to a
different library.
