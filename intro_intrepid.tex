% Ellen
\chapter{Intrepid}
Intrepid is a C++ library developed as part of Sandia's Trilinos Project,
providing algebraic operations over multi-dimensional arrays.  Tensor
contractions are one class of tools implemented by Intrepid, widely used in
high-performance simulation software.

One particular type of tensor contraction, two-dimensional matrix
multiplication, is known to show great speedup when implemented using CPU and
especially GPU parallelism.  For this reason among others, the team considered
Intrepid tensor contractions to have good theoretical potential to derive
significant performance gains from parallelism.

\section{Tensor Contractions}
% TODO

\section{Intrepid Contractions Overview}
Intrepid provides nine types of tensor contraction, differing by input
dimensionality, number of indices contracted away, and output dimensionality.
It is most simple to classify Intrepid's tensor contractions by number of
indices contracted away and output dimensionality.

Intrepid tensor contractions contract away one, two, or three indices.  The
output of a single contraction can be a scalar (zero-dimensional), a vector
(one-dimensional), or a matrix (two-dimensional).  Each combination of number of
contraction indices and output dimension is handled by one Intrepid tensor
contraction kernel.

\begin{table}[ht]
        \begin{tabular} {| l | l | l | l |l |}
            \hline
            \textbf{Kernel Name} & \textbf{Left Input} & \textbf{Right Input} &
            \textbf{Output} & \textbf{Contraction Indices}\\
            \hline
            ContractDataDataScalar   & 1D & 1D & Scalar & One \\
            ContractDataDataVector   & 2D & 2D & Scalar & Two \\
            ContractDataDataTensor   & 3D & 3D & Scalar & Three \\
            ContractDataFieldScalar  & 2D & 1D & Vector & One \\
            ContractDataFieldVector  & 3D & 2D & Vector & Two \\
            ContractDataFieldTensor  & 4D & 3D & Vector & Three \\
            ContractFieldFieldScalar & 2D & 2D & Matrix & One \\
            ContractFieldFieldVector & 3D & 3D & Matrix & Two \\
            ContractFieldFieldTensor & 4D & 4D & Matrix & Three \\
            \hline
        \end{tabular}
\caption{Summary of the nine Intrepid tensor contraction kernels
\label{tab:IntrepidContractionSummary}}
\end{table}

In order to more easily discuss specific tensor contraction kernels in Intrepid,
it helps to first understand the naming convention used for the kernel names.
Each kernel's name contains two suffixes, where the first indicates the
dimensionality of the output and the second indicates the number of contraction
indices.

\begin{table}[ht]
    \begin{center}
        \begin{tabular} {| l | l | l |}
            \hline
            \textbf{String} & \textbf{Position} & \textbf{Meaning} \\
            \hline
            DataData    & First Suffix  & Scalar Output    \\
            DataField   & First Suffix  & Vector Output    \\
            FieldField  & First Suffix  & Matrix Output    \\
            Scalar      & Second Suffix & One Contraction Index \\
            Vector      & Second Suffix & Two Contraction Indices \\
            Tensor      & Second Suffix & Three Contraction Indices \\
            \hline
        \end{tabular}
    \end{center}
\caption{Intrepid tensor contraction suffixes
\label{tab:IntrepidNamingConvention}}
\end{table}

For instance, the first suffix \texttt{DataData} is used for kernels that
produce scalar outputs, and the second suffix \texttt{Scalar} is used for
kernels that contract away one dimension.  Therefore, the Intrepid kernel
\texttt{ContractDataDataScalar} produces scalar outputs and contracts away one
dimension, and so by necessity the inputs for a single contraction must be
vectors.  All of the Intrepid tensor contraction kernel suffixes are summarized
in Table~\ref{tab:IntrepidNamingConvention}.  The nine tensor contractions in
Intrepid are summarized in Table~\ref{tab:IntrepidContractionSummary}.

Note that the tensor contraction kernels in Intrepid each actually performs many
contractions; for instance, \texttt{ContractDataDataScalar}, which performs
contractions of two input vectors to a single output scalar (dot product),
actually calculates an array of dot products.  That is, the inputs are both
arrays of vectors, and the output is an array of scalars.  This is represented
in the code using a dummy index, which we call the \texttt{Cell} index.

\section{ContractDataDataScalar}
\texttt{ContractDataDataScalar} is the simplest tensor contraction in Intrepid.
The kernel takes two arrays of vectors and outputs an array of scalars. A
snippet showing the simple serial implementation of
\texttt{ContractDataDataScalar} can be seen in
Figure~\ref{lst:ContractDataDataScalarSerial}.

\begin{figure}[ht]
    \begin{lstlisting}
    for (int c = 0; c < numCells; ++c) {
        for (int qp = 0; qp < quadraturePoints; ++qp) {
            output[c] += leftInput[c][qp] * rightInput[c][qp];
        }
    }
    \end{lstlisting}
\caption{Code from serial \texttt{ContractDataDataScalar}
\label{lst:ContractDataDataScalarSerial}} 
\end{figure}

As shown in 
Figure~\ref{lst:ContractDataDataScalarSerial}, this kernel contracts away the
\texttt{Quadrature Points} dimension, leaving only the \texttt{Cell} dimension 

\section{ContractDataDataVector}
% TODO
% DataDataVector = (c, p, t)      * (c, p, t)      -> (c)

\section{ContractDataDataTensor}
% TODO
% DataDataTensor = (c, p, t1, t2) * (c, p, t1, t2) -> (c)



\section{ContractDataFieldScalar}
% TODO
% DataFieldScalar = (c, l, p)         * (c, p)         -> (c, l)

\section{ContractDataFieldVector}
% TODO
% DataFieldVector = (c, l, p, t)      * (c, p, t)      -> (c, l)

\section{ContractDataFieldTensor}
% TODO
% DataFieldTensor = (c, l, p, t1, t2) * (c, p, t1, t2) -> (c, l)



\section{ContractFieldFieldScalar}
% TODO
% FieldFieldScalar = (c, l, p)         * (c, r, p)         -> (c, l, r)

\section{ContractFieldFieldVector}
% TODO
% FieldFieldVector = (c, l, p, t)      * (c, l, p, t)      -> (c, l, r)

\section{ContractFieldFieldTensor}
% TODO
% FieldFieldTensor = (c, l, p, t1, t2) * (c, r, p, t1, t2) -> (c, l, r)

