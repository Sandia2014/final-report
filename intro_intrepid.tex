% Ellen
\chapter{Intrepid}
Intrepid is a C++ library developed as part of Sandia's Trilinos Project,
providing algebraic operations over multi-dimensional arrays.  Tensor
contractions are one class of tools implemented by Intrepid, widely used in
high-performance simulation software.

One particular type of tensor contraction, two-dimensional matrix
multiplication, is known to show great speedup when implemented using CPU and
especially GPU parallelism.  For this reason among others, the team considered
Intrepid tensor contractions to have good theoretical potential to derive
significant performance gains from parallelism.

\section{Tensor Contractions}
A tensor can be thought of as a multidimensional array.  Tensor contractions are
algebraic operations over tensors in which pairs of indices, one from each of
the two input tensors, are ``contracted'' together, reducing the dimensionality
of the output tensor.  

For instance, two matrices (two-dimensional tensors), can have no indices
contracted away, yielding a four-dimensional outer product.  If the size of one
dimension on the left matrix matches the size of one dimension on the right
matrix, that dimension can be contracted by summing the products of the left and
right matrix entries for each element in the contraction index.  This is
standard matrix multiplication.  If the two matrices are of identical size, then
both indices can be contracted away in an inner product, each element in the
left matrix multiplied pairwise with the corresponding element int the right
matrix and summed to yield a scalar output.

Tensor contractions can be generalized to higher-dimension tensors, such as
three or four-dimensional tensors.

\section{Intrepid Contractions Overview}
Intrepid provides nine types of tensor contraction, differing by input
dimensionality, number of indices contracted away, and output dimensionality.
It is most simple to classify Intrepid's tensor contractions by number of
indices contracted away and output dimensionality.

Intrepid tensor contractions contract away one, two, or three indices.  The
output of a single contraction can be a scalar (zero-dimensional), a vector
(one-dimensional), or a matrix (two-dimensional).  Each combination of number of
contraction indices and output dimension is handled by one Intrepid tensor
contraction kernel.

\begin{table}[ht]
        \begin{tabular} {| l | l | l | l |l |}
            \hline
            \textbf{Kernel Name} & \textbf{Left Input} & \textbf{Right Input} &
            \textbf{Output} & \textbf{Contraction Indices}\\
            \hline
            ContractDataDataScalar   & 1D & 1D & Scalar & One \\
            ContractDataDataVector   & 2D & 2D & Scalar & Two \\
            ContractDataDataTensor   & 3D & 3D & Scalar & Three \\
            ContractDataFieldScalar  & 2D & 1D & Vector & One \\
            ContractDataFieldVector  & 3D & 2D & Vector & Two \\
            ContractDataFieldTensor  & 4D & 3D & Vector & Three \\
            ContractFieldFieldScalar & 2D & 2D & Matrix & One \\
            ContractFieldFieldVector & 3D & 3D & Matrix & Two \\
            ContractFieldFieldTensor & 4D & 4D & Matrix & Three \\
            \hline
        \end{tabular}
\caption{Summary of the nine Intrepid tensor contraction kernels
\label{tab:IntrepidContractionSummary}}
\end{table}

In order to more easily discuss specific tensor contraction kernels in Intrepid,
it helps to first understand the naming convention used for the kernel names.
Each kernel's name contains two suffixes, where the first indicates the
dimensionality of the output and the second indicates the number of contraction
indices.

\begin{table}[ht]
    \begin{center}
        \begin{tabular} {| l | l | l |}
            \hline
            \textbf{String} & \textbf{Position} & \textbf{Meaning} \\
            \hline
            DataData    & First Suffix  & Scalar Output    \\
            DataField   & First Suffix  & Vector Output    \\
            FieldField  & First Suffix  & Matrix Output    \\
            Scalar      & Second Suffix & One Contraction Index \\
            Vector      & Second Suffix & Two Contraction Indices \\
            Tensor      & Second Suffix & Three Contraction Indices \\
            \hline
        \end{tabular}
    \end{center}
\caption{Intrepid tensor contraction suffixes
\label{tab:IntrepidNamingConvention}}
\end{table}

For instance, the first suffix \texttt{DataData} is used for kernels that
produce scalar outputs, and the second suffix \texttt{Scalar} is used for
kernels that contract away one dimension.  Therefore, the Intrepid kernel
\texttt{ContractDataDataScalar} produces scalar outputs and contracts away one
dimension, and so by necessity the inputs for a single contraction must be
vectors.  All of the Intrepid tensor contraction kernel suffixes are summarized
in Table~\ref{tab:IntrepidNamingConvention}.  The nine tensor contractions in
Intrepid are summarized in Table~\ref{tab:IntrepidContractionSummary}.

Note that the tensor contraction kernels in Intrepid each actually performs many
contractions; for instance, \texttt{ContractDataDataScalar}, which performs
contractions of two input vectors to a single output scalar (dot product),
actually calculates an array of dot products.  That is, the inputs are both
arrays of vectors, and the output is an array of scalars.  This is represented
in the code using a dummy index, which we call the \texttt{Cell} index.

\section{ContractDataDataScalar}
\texttt{ContractDataDataScalar} is the simplest tensor contraction in Intrepid.
The kernel takes two arrays of vectors and outputs an array of scalars. A
snippet showing the simple serial implementation of
\texttt{ContractDataDataScalar} can be seen in
Figure~\ref{lst:ContractDataDataScalarSerial}.

\begin{figure}[ht]
    \begin{lstlisting}
    for (int c = 0; c < numCells; ++c) {
      for (int qp = 0; qp < quadraturePoints; ++qp) {
        output[c] += leftInput[c][qp] * rightInput[c][qp];
      }
    }
    \end{lstlisting}
\caption{Code from serial \texttt{ContractDataDataScalar}
\label{lst:ContractDataDataScalarSerial}} 
\end{figure}

As shown in 
Figure~\ref{lst:ContractDataDataScalarSerial}, this kernel contracts away the
\texttt{Quadrature Points} dimension, leaving only the \texttt{Cell} dimension.
It can also be thought of as an array of dot products.

\section{ContractDataDataVector}
\texttt{ContractDataDataVector} takes two arrays of two-dimensional tensors
(matrices) and computes an array of their inner products.
\begin{figure}[ht]
    \begin{lstlisting}
    for (int c = 0; c < numCells; ++c) {
      for (int qp = 0; qp < quadraturePoints; ++qp) {
        for (int t = 0; t < iVec; ++t) {
          output[c] += leftInput[c][qp][t] * rightInput[c][qp][t];
        }
      }
    }
    \end{lstlisting}
\caption{Code from serial \texttt{ContractDataDataVector}
\label{lst:ContractDataDataVectorSerial}} 
\end{figure}

As shown in Figure~\ref{lst:ContractDataDataVectorSerial},
\texttt{ContractDataDataVector} is very similar to
\texttt{ContractDataDataScalar}, except it contracts two indices instead of
one.

\section{ContractDataDataTensor}\label{section:ContractDataDataTensor}
\texttt{ContractDataDataTensor} takes two arrays of three-dimensional tensors
and computes an array of their inner products.

\begin{figure}[ht]
    \begin{lstlisting}
    for (int c = 0; c < numCells; ++c) {
      for (int qp = 0; qp < quadraturePoints; ++qp) {
        for (int t1 = 0; t1 < iVec1; ++t1) {
          for (int t2 = 0; t2 < iVec2; ++t2) {
            output[c] += leftInput[c][qp][t1][t2] * 
                         rightInput[c][qp][t1][t2];
          }
        }
      }
    }
    \end{lstlisting}
\caption{Code from serial \texttt{ContractDataDataTensor}
\label{lst:ContractDataDataTensorSerial}} 
\end{figure}

As shown in Figure~\ref{lst:ContractDataDataTensorSerial},
\texttt{ContractDataDataTensor} is very similar to\\
\texttt{ContractDataDataVector} and \texttt{ContractDataDataScalar}, but has
three contraction indices.

\section{ContractDataFieldScalar}

\texttt{ContractDataFieldScalar} takes an array of matrices and an array of
vectors and contracts away one index.

\begin{figure}[ht]
    \begin{lstlisting}
    for (int c = 0; c < numcells; ++c) {
      for (int l = 0; l < lbf; ++l) {
        for (int qp = 0; qp < quadraturepoints; ++qp) {
          output[c][l] += left[c][l][qp] * right[c][qp];
        }
      }
    }
    \end{lstlisting}
\caption{Code from serial \texttt{ContractDataFieldScalar}
\label{lst:ContractDataFieldScalarSerial}} 
\end{figure}

As shown in Figure~\ref{lst:ContractDataFieldScalarSerial},
\texttt{ContractDataFieldScalar} has two non-contraction indices, the
\texttt{Left Basis Function} index and the dummy \texttt{Cell} index.  The
output for this contraction is therefore an array of vectors instead of an array
of scalars.

\section{ContractDataFieldVector}
\texttt{ContractDataFieldVector} takes an array of three-dimensional tensors and
an array of vectors, and contracts away two indices to produce an array of
vectors.

\begin{figure}[ht]
    \begin{lstlisting}
    for (int c = 0; c < numcells; ++c) {
      for (int l = 0; l < lbf; ++l) {
        for (int qp = 0; qp < quadraturepoints; ++qp) {
          for (int t = 0; t < iVec; ++t) {
            output[c][l] += left[c][l][qp][t] * right[c][qp][t];
          }
        }
      }
    }
    \end{lstlisting}
\caption{Code from serial \texttt{ContractDataFieldVector}
\label{lst:ContractDataFieldVectorSerial}} 
\end{figure}

As shown in Figure~\ref{lst:ContractDataFieldVectorSerial},
\texttt{ContractDataFieldVector} is similar to \texttt{ContractDataFieldScalar},
but has two contraction indices. 

\section{ContractDataFieldTensor}
\texttt{ContractDataFieldTensor} takes an array of four-dimensional tensors and
an array of three-dimensional tensors, and contracts away two indices to produce
an array of vectors.

\begin{figure}[ht]
    \begin{lstlisting}
    for (int c = 0; c < numcells; ++c) {
      for (int l = 0; l < lbf; ++l) {
        for (int qp = 0; qp < quadraturepoints; ++qp) {
          for (int t1 = 0; t1 < iVec1; ++t1) {
            for (int t2 = 0; t2 < iVec2; ++t2) {
              output[c][l] += left[c][l][qp][t1][t2] * right[c][qp][t1][t2];
            }
          }
        }
      }
    }
    \end{lstlisting}
\caption{Code from serial \texttt{ContractDataFieldTensor}
\label{lst:ContractDataFieldTensorSerial}} 
\end{figure}

As shown in Figure~\ref{lst:ContractDataFieldTensorSerial},
\texttt{ContractDataFieldTensor} is similar to \texttt{ContractDataFieldScalar},
but has three contraction indices. 

\section{ContractFieldFieldScalar}
\texttt{ContractFieldFieldScalar} takes in two arrays of matrices and performs
matrix multiplication on each element, yielding an output array of matrices.

\begin{figure}[ht]
    \begin{lstlisting}
    for (int c = 0; c < numcells; ++c) {
      for (int l = 0; l < lbf; ++l) {
        for (int r = 0; r < rbf; ++r) {
          for (int qp = 0; qp < quadraturepoints; ++qp) {
            output[c][l][r] += left[c][l][qp] * right[c][r][qp];
          }
        }
      }
    }
    \end{lstlisting}
\caption{Code from serial \texttt{ContractFieldFieldScalar}
\label{lst:ContractFieldFieldScalarSerial}} 
\end{figure}

As shown in Figure~\ref{lst:ContractFieldFieldScalarSerial},
\texttt{ContractFieldFieldScalar} has two non-contraction indices, so the output
is an array of matrices.

\section{ContractFieldFieldVector}
\texttt{ContractFieldFieldVector} takes two arrays of three-dimensional tensors
and contracts away two indices, keeping the \texttt{Cell} dummy dimension as
well as the left and right basis function indices.

\begin{figure}[ht]
    \begin{lstlisting}
    for (int c = 0; c < numcells; ++c) {
      for (int l = 0; l < lbf; ++l) {
        for (int r = 0; r < rbf; ++r) {
          for (int qp = 0; qp < quadraturepoints; ++qp) {
            for (int t = 0; t < iVec; ++t) {
              output[c][l][r] += left[c][l][qp][t] * right[c][r][qp][t];
            }
          }
        }
      }
    }
    \end{lstlisting}
\caption{Code from serial \texttt{ContractFieldFieldVector}
\label{lst:ContractFieldFieldVectorSerial}} 
\end{figure}

As shown in Figure~\ref{lst:ContractFieldFieldVectorSerial},
\texttt{ContractFieldFieldVector} is similar to
\texttt{ContractFieldFieldScalar}, but with two contraction indices.

\section{ContractFieldFieldTensor}
\texttt{ContractFieldFieldTensor} is the most complex of the tensor contraction
kernels in Intrepid.  This kernel takes two four-dimensional tensors and
and contracts away three indices, keeping the \texttt{Cell} dummy dimension as
well as the left and right basis function indices.

\begin{figure}[ht]
    \begin{lstlisting}
    for (int c = 0; c < numcells; ++c) {
      for (int l = 0; l < lbf; ++l) {
        for (int r = 0; r < rbf; ++r) {
          for (int qp = 0; qp < quadraturepoints; ++qp) {
            for (int t1 = 0; t1 < iVec1; ++t1) {
              for (int t2 = 0; t2 < iVec2; ++t2) {
                  output[c][l][r] += left[c][l][r][qp][t1][t2] *
                  right[c][l][r][qp][t1][t2];
              }
            }
          }
        }
      }
    }
    \end{lstlisting}
\caption{Code from serial \texttt{ContractFieldFieldTensor}
\label{lst:ContractFieldFieldTensorSerial}} 
\end{figure}

As shown in Figure~\ref{lst:ContractFieldFieldTensorSerial},
\texttt{ContractFieldFieldTensor} is similar to
\texttt{ContractFieldFieldScalar}, but with two contraction indices.

